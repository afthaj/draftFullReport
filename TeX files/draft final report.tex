%!TEX TS-program = xelatex
%!TEX encoding = UTF-8 Unicode
\documentclass[12pt, oneside]{report}

\bibliographystyle{plainnat}

\usepackage{geometry}
\geometry{a4paper}
%\geometry{landscape}
%\usepackage[parfill]{parskip}
\usepackage{graphicx}
\usepackage{amssymb}
\usepackage{url}
\usepackage[authoryear, round]{natbib}
\usepackage[table]{xcolor}
\usepackage{color, colortbl}

\title{
A Decision Support System
\\ for Public Transportation
\\ in Developing Countries
}

\author{Aftha Jaldin}
\date{11th of October 2013}

\begin{document}
\maketitle

\pagenumbering{arabic}
\tableofcontents
\newpage
\pagenumbering{arabic}

\listoffigures
\newpage

\listoftables
\newpage

\section*{Abstract}

\paragraph{ } The Private Bus Passenger Transportation System is the most widely used Public Transportation System in Sri Lanka. It is used as the primary transport method for passengers within Colombo, the most populous city in the country, as well as between Colombo and other cities as well as suburbs.

However, the system is rife with inefficiencies and shortcomings. Delays, overcrowded buses and bus strikes are commonplace and the service is poor to say the least. The problems in the system do not merely affect the transportation industry but has a knock-on effect to all walks of daily life for the people in the country. The productivity of the workforce suffers due to this inefficiency in the main public transport system.

The biggest problem that the commuters outline is the need to improve the service in terms of its schedule reliability and passenger overcrowding. Meanwhile the bus operators complain that the revenue leakage is a major issue and needs to be rectified. However, the status quo continues without any solutions and these problems continue to haunt the average commuter.

Therefore, this research attempts to analyse the problems prevalent in the system and provide a bus scheduling solution to minimize/eliminate the existing problems associated with the inefficient bus service.

\newpage

\section*{Acknowledgement}
I would like to take this opportunity to thank the following people who have helped me thus far in my final year research project.
\begin{itemize}
\item Dr. Shiromi Arunatilake, for acting as my supervisor, overseeing my work and giving her valuable input into the thinking process of the project.
\item Dr. Jeewani Goonathilake, for providing her assistance whenever possible in her capacity as the Course Coordinator for the ICT Research Projects.
\item Mr. Pradeep Fernando, the Head of the GPS Tracking and Monitoring Unit of the National Transport Commission for providing assistance in gathering information on the GPS tracking system which has been deployed on inter-provincial NTC buses.
\item Mr. Dhanushka of the GPS Tracking and Monitoring Unit of the National Transport Commission for providing information regarding the existing tracking system.
\item Mr. Muditha Navaratne from the Timetable Unit of the National Transport Commission.
\item Dr. Chaminda Ranasinghe, Chief Executive Officer at IdeaHub (Pvt.) Ltd for providing his valuable insights.
\item Mr. K.A.R.A. Ranjith, the Operations Manager of the Western Province Passenger Transport Authority in providing information regarding the private bus service in the province.
\item Prof. Amal Kumarage, a Senior Lecturer of the Department of Transport and Logistics Management at the University of Moratuwa.
\item Mr. Janaka Weerawardana, doctoral student at the University of Moratuwa, Department of Transport and Logistics Management.
\item My Parents for supporting me throughout the formulation of this report.
\end{itemize}


\chapter{Introduction}
\label{chapter-Introduction}

\section{Motivation}
\label{section-Motivation}

\paragraph{ } The Private Bus Transportation System is a famously contentious topic in Sri Lanka. For the average commuter, complaining about the system is part and parcel of everyday life. As a commuter myself and having used public transportation extensively for more than a decade, I have wondered many a time about the reasons why the bus service is in this state and why people keep complaining about it so very vehemently. The constant complaints are justified, as the bus service is grossly inefficient and lack proper service quality. The inefficiency in the system leads to lost productivity and dissatisfaction by the commuters. In the end it is the country that suffers from not having an efficient public transportation system for the people.

It is in this background that I was motivated to carry out a research regarding the Private Bus Service in the Western Province. The objective was to pinpoint the existing problems and possibly provide an IT solution for them. During the course of the research and interviews with the numerous people responsible, it occurred to me that the system has a myriad of things wrong with it. 

Careful management was the only requirement to overcome some problems, while others required a restructuring of the system, and still others required just an implementation of a solution and regulation of the service properly. Issues where research was required also existed. As I was passionate about the problem, I was intent on conducting a research to find solutions to the existing problems.

\newpage

\section{Background Study}
\label{section-BackgroundStudy}

\paragraph{ } Bus Passenger Transport is the highest and most widely used mode of public transport in Sri Lanka. The system is divided into Inter-Provincial and Intra-Provincial bus services. The service is operated by privately run (private buses) as well as state-run (SLTB) buses. The Sri Lanka Transport Board manages the SLTB buses and their jurisdiction is island wide regardless of the district or province. However, the organizational structure of private buses is different. All Inter-Provincial Private Buses are governed by the National Transport Commission (NTC) while the private bus services within each of Sri Lanka’s provinces are governed by the Passenger Transport Authorities of the respective provinces. Accordingly, the Western Province Road Passenger Transport Authority (WP RPTA) handles the governance of the Private Bus Passenger Transport System in the Western Province.

The major difference between the system in use in Sri Lanka and other countries is that the owners and operators of the buses are independent contractors and not the Central Government or the Provincial Authority. This has led to numerous problems and we shall discuss them in detail during the course of this document.

Let us take a look at the statistics involved with the bus service in Sri Lanka. According to the Draft National Policy on Transport in Sri Lanka \citep{MinistryofTransport2008}, public transport accounts for nearly 73\% of the total motorized passenger transport in the country. It also serves as the only means of transport for the majority of the population.

Of this, bus transportation accounts for nearly 68\% (a 93\% share in the public transport sphere), while rail transport accounts for the remaining 5\% (a 7\% share in the public transport sphere) \citep{MinistryofTransport2008}.

Within the bus transportation system, state-run bus services account for a 23\% share (about one-third of the share of bus transport) while private operators have a share of 45\% (a two-thirds share) provided by small-scale operators.

\begin {figure} [h!]
\centering
\includegraphics [scale=0.7] {totalMotorTransportPieChart}
\caption [Total Motorized Passenger Transport in SL] {Detailed View of Total Motorized Passenger Transport in Sri Lanka. Source: \citep{MinistryofTransport2008}}
\label {image-totalMotorTransportPieChart}
\end {figure}

Taking a look at the state-run bus service, data obtained from the Sri Lanka Transport Board show that approximately 2.5 million passengers island wide commute daily on close to 4500 SLTB buses \citep{SriLankaTransportBoard2010}. These buses travel an average of 2000km a day \citep{SriLankaTransportBoard2012}. 

Considering the buses operated by private bus operators, it is estimated that 10 million commuters travel daily on approximately 18,000 private buses currently in use in the country \citep{Silva2010}. Data gathered from the WP RPTA show that there are around 7000 private buses servicing the Western Province and its’ routes which is close to 450 in number. 

Looking at these statistics, we clearly see that the buses run by the private bus operators are far more in number compared to the state-run buses. It also shows that the majority of the commuters rely on the buses operated by private bus operators. These buses function as small independent operators and what is astounding is the fact that in 30 years of operation, not even a single major private bus operator has emerged. The private bus “cartel” is highly unionized and dictates terms to the commuters as well as the government on a regular basis \citep{AdaDerana2012}. Bus strikes are very common and when they do happen, the commuters are placed in great discomfort \citep{Samarajiva2012}, \citep{ColomboPage2012}.

In contrast, the state run bus service has a governing body, the SLTB, but it has been in steady decline since the 1970’s owing to mismanagement and cost overruns \citep{AnswersDotCom2012}. It has now become a money-sucking state entity and continues to waste the tax payer's money with no solution in the horizon \citep{LBO2011}, \citep{Sirimanne2013}. The number of SLTB buses in daily operation is well below the number of private buses and because of this, the state-run bus service is unreliable. Presently, it is but a teardrop in a sea of public transport dominated by the private bus operators. For example, as pointed out previously, there are more private buses in the western province (around 7000) than there are SLTB buses in the entire island (around 4500).Given that the average commuter cannot and does not wait for hours on end till an SLTB bus arrives, they have no other option but to use the private bus service. \citep{Wijayapala2012}, \citep{Azwer2012}

\subsection{Justification of Research Project}

\paragraph{} As mentioned earlier, the Private Bus service is the largest and most widely used Passenger Transport system not only in the Western Province but also in Sri Lanka. However, commuters are constantly dissatisfied with the service provided and there seems to be no alternative. The railway system has its own problems and inefficiencies and a solution to that demands separate research.

According to World Bank statistics, Sri Lanka’s population currently stands at 20.869 million people \citep{WorldBank2013}. Being the most densely populated, the Western Province has a population of 5.8 million people \citep{DepartmentofCensusandStatistics2012}, which is 27.8\% of the total population. This means that more than a quarter of the people in Sri Lanka live and commute in the Western Province. Therefore, it is clearly evident that an improvement in the level of service is needed seeing as it affects more than a quarter of the country’s population.

As mentioned earlier, there are around 7000 private buses servicing the Western Province. To put that into perspective, the SLTB only has a cadre of 4500 buses islandwide. This means that there are more private buses in the Western Province than there are SLTB buses in the entire country. This shows that it is a large transportation system that affects more than a quarter of the population in the country on a daily basis.

Also of importance to note is that the number of complaints related to the Inter-Provincial Private Bus Service has doubled this year compared to last year. Chairman of the NTC is reported as saying that over 40 complaints are received per day and this number is increasing \citep{Wickremasekara2012}, \citep{Range2012}. This means that the current strategy of scheduling buses is clearly not working properly which is leading to increased commuter dissatisfaction and complaints.

Therefore, research into the private bus passenger transport system is essential. It is a large system that affects a large amount of people on a daily basis and the service needs to improve significantly in order for this country to be more productive, the people to be satisfied and for the country to be more attractive to tourists.

\newpage

\section{Existing Problems}
\label{section-ExistingProblems}
There are several issues that commuters highlight consistently. These are listed below. 
\begin {enumerate}
\item Loitering of buses at bus stops
\item Slowness of the buses
\item Unreliability of the bus schedule
\item Overcrowding of buses
\item Discourteous service by the bus operators (conductors as well as drivers)
\item Overcharging of bus fares (commuters complained mainly about not receiving the proper change money back from the conductor)
\item Cleanliness and general presentability of bus operators (conductors as well as drivers)
\item Lack of information (regarding the rules and regulations for the bus operators as well as the commuters, bus schedules, fare tables etc.)
\item Other issues pointed out consistently by commuters
	\begin {enumerate}
	\item Non-issuance of tickets
	\item Neglect of road rules
	\item Wreckless driving
	\item Diverting attention to other activities while driving (texting, talking on the phone, eating, drinking etc)
	\item Picking up and dropping off commuters at undesignated bus stops
	\item The operators (driver and conductor) don't wear the proper attire (the designated uniform)
	\item Failure to display the fare table to the commuters
	\item Operating without a valid driver's license, revenue license, and/or route permit
	\end {enumerate}
\end {enumerate}
References: \citep{Wickremasekara2012}, \citep{Range2012}, comments by frequent commuters, data gathered through a survey.

\newpage

\section{Reasons for the Problems}
Each of the above mentioned problems have reasons that may cause them. Let us take a look at some possible reasons.
\begin {enumerate}
\item Loitering, slowness and unreliability of the buses. (Points 1,2 and 3 above)
\begin {itemize}
\item These are mainly caused by bus operators failing to follow the proper timetable.
\item The time table itself could be at fault as it may allow the bus driver to go at a snail’s pace and be tardy.
\item The traffic situation in urban roads is also to blame for these 2 problems.
\item Delays are also caused by the lack of buses in a particular route. Again, this boils down to the inefficiency and ineffectiveness of the timetable.
\end {itemize}
\item Overcrowding of buses (Point 4 above)
\begin {itemize}
\item On one hand the bus operators are at fault for trying to maximize their profits by overcrowding the buses and disregarding passenger comfort and safety. On the other, the commuters are at fault for getting into an already crowded bus. The commuters feel that if they do not get into the bus right now, they might be late to their destination by waiting for another to come along. This may be caused by a lack of trust in the bus schedule so in effect this problem is caused by the timetables once again as pointed out previously.
\item It is important to note that the authorities allow this overcrowding to happen as they have no rules governing the crowds in buses.
\end {itemize}
\item Discourteous service, overcharging of bus fares and cleanliness \& general presentability of bus operators (Points 5, 6 and 7 above)
\begin {itemize}
\item The commuters frequently complain about these 3 points and the bus operators are solely to blame. They are a result of improper training and a lack of respect and professionalism in the industry.
\end {itemize}
\item Lack of information (Point 8 above)
\begin {itemize}
\item The blame for this lies with the bus operators as well as the regulatory body, the Western Province Road Passenger Transport Authority. The WP RPTA does not have a proper channel (website, information booklet etc.) to display and disseminate information regarding bus timetables and the rules and regulations governing the bus services etc. It is compulsory and of the utmost importance that this and other information is displayed to the pubic freely but it is not being done properly leading to the public frequently complaining and being misinformed.
\end {itemize}
\end {enumerate}

\newpage

\section{Current Bus Scheduling Process in Sri Lanka}
\label{section-CurrentBusSchedulingProcessInSriLanka}

\paragraph{ } Bus scheduling and timetabling in Sri Lanka (both inter and intra-provincial) has been a manual process in the past and it continues to be in the present day. Although IT tools can be used for scheduling and timetabling, the authorities still use an age-old manual process of scheduling buses for the bus routes (Data gathered through interviews with the Scheduling Unit of the WP RPTA).

Up until the mid 2000’s, the bus routes in the Western Province did not have proper bus schedules. The buses were dispatched from the terminals in the order and frequency they arrived. No scientific methodology was used in identifying headways, slack times or schedules. However, following the research efforts from the University of Moratuwa and Mr. Anuradha Piyadasa in particular, bus schedules were implemented, circa 2005. The methodology used in formulating these schedules is explored later in this section. These schedules were then agreed upon and put into circulation in the Western Province bus routes. Although, the formulation of these schedules was done using a software system (developed by Mr. Piyadasa), this system is not being used at the moment. The main steps of the process are illustrated in Figure~\ref{image-currentBusSchedulingProcess}.

\begin{figure}[h!]
\centering
\includegraphics [scale=0.7] {currentBusSchedulingProcess}
\caption[Current Bus Scheduling Process in SL]{Current Bus Scheduling Process in Sri Lanka. Source: \citep{Piyadasa2005}}
\label{image-currentBusSchedulingProcess}
\end{figure}

This process of bus scheduling is an extension of the process put forth by Dennis Huisman in 2004 \citep{Piyadasa2005}. The first step of the process is to conduct a survey. This identifies the existing passenger demand, the quality of the service the buses provide on a given route, the requirements of the regulatory body and any special characteristics of a given route. 

The surveyors ride the buses at variously chosen days and times (to and from the main terminal), note down the trip times and identify the delays and loitering points of the buses on the given route. They also gauge the passenger demand by noting how crowded the bus gets at different stages of the trip. This is a rough estimate of the demand, as the surveyors do not have access to the exact quantitative data.

The next step is the timetable. They formulate this by calculating the average headway for the route. During the formulation of the timetable, the schedulers determine the time the bus service commences and terminates daily for the route. This is done through analyzing the passenger demand data they gather. It is important to correctly identify the time the service commences and terminates daily as it affects the revenue gained by the bus operators and may lead to the operators being unhappy with the timetable. If the service starts too early in the morning or ends too late at night the operators will run at a loss and will not be able to provide a proper service.

Next is the scheduling of the buses to the timetable. The scheduling officers schedule the buses to the bus routes manually using their observations, experience and knowledge. Once the scheduling is complete, the crew rosters are formulated using the schedules. 

Finally, the WP RPTA, the Bus Owners, the Bus Operators and the Union reps then agree on the revised bus schedule. After the agreement is signed, the revised schedule is placed into circulation and used on the bus route (Data gathered through interviews with the Scheduling Unit of the WP RPTA).
As the crew of a particular bus only operates that bus, the steps of vehicle scheduling and crew scheduling could be and should be carried out together. This is known as Integrated Vehicle and Crew Scheduling and literature to support this has been mentioned in the Literature Review of this thesis document.

As mentioned previously, until the mid-2000’s the private buses in the Western Province did not have predefined schedules to operate with. The buses were dispatched as soon as they came in to the terminal and the operators did not have predefined working hours and trips to complete for each working day \citep{Piyadasa2005}. However, thanks to research efforts by the University of Moratuwa almost all of the 450+ routes in the Western Province now have working timetables. Despite these timetables being implemented and in operation, they are not being properly adhered to and the quality of the bus service is still well below what is needed.

The bus schedules that are currently in operation take into account both the Economic and Financial cost of the service to the country, the commuters and the bus operators and optimizes the dispatching of the buses via optimal headway manipulation \citep{Piyadasa2005}. After doing numerous mathematical calculations, the methodology identifies the optimal average headway for the route. After identifying the headway, the available buses are scheduled to fill up the timetable as optimally as possible.

A software system was developed to be used to calculate the headways. Ironically however, the system is not being used by the WP RPTA to which it was developed. Instead they use a manual method of determining headway. The NTC however, uses the system to schedule the inter-provincial bus system (data gathered through interviews with Mr. Anuradha Piyadasa, the researcher and developer of the software system).

On inquiry from the personnel at the Scheduling Unit of the WP RPTA, they said that a common observation was that there is a plethora of buses for many of the bus routes in the province, which leads to scheduling difficulties for the Authority. This is due to previous regimes simply doing surveys on the bus service and adding additional buses to the roads, regardless of the requirement. Bus route permits have been issued in mass numbers as a stop-gap solution to the ailing bus service without accounting for the effect it will have on the system as well as the traffic situation. This leads to the revenue gained by each individual bus gradually decreasing (The number of passengers aka the demand stays fairly constant while the number of buses increases which means there are more buses to share in the same revenue. This means that the revenue gained by each individual bus decreases). This has and continues to hinder any efforts for proper reform, restructuring and reengineering of the private bus service in the country.

The timetables that have been implemented currently have research backing them but the bus operators have found numerous ways to circumvent the objectives of these schedules which is to ensure a timely and efficient bus service. This became clearly evident during discussions with the Scheduling Unit of the WP RPTA and from the constant dissatisfaction by the commuters. The bus operators consistently look to maximize their profits with a disregard for the level of service offered to the passenger. The timekeepers and Officers-In-Charge at bus terminals also add to the problem by accepting bribes and neglecting their duties.

Obvious solutions to these problems would be to reduce the trip times in the schedules even more and implement tougher regulations. However, trip times can only be reduced up to a certain point before they become impracticable. The Scheduling Unit is revising the schedules at the moment but it is moving at a snail’s pace, not unlike the buses they schedule. 

Also of note is how the revision process is handled. A revision to an existing schedule is not done unless and until sufficient complaints are received from the commuters. One would then think that there was a reasonably efficient complaint process and system in place but this is not so. The complaint mechanism is non-structured which leads to a perfect storm of problems. The commuters are not made aware of the complaint mechanism (no proper channel of disseminating information to the public by way of website/mobile app etc) and complaints are not received, documented and acted upon in a proper fashion (At times only the security guard at the scheduling unit is available to answer the phone and take down complaints when they are received) leading to the status quo continuing. (Source: Discussions with the Scheduling Personnel at the WP RPTA)

\newpage

\section{Research Goals and Objectives}
\label{section-ResearchGoalsAndObjectives}

\paragraph{ } Of the problems mentioned in Section~\ref{section-ExistingProblems}, implementing a better bus schedule could solve the first 3. The problem of overcrowded buses could also logically be solved through implementing an effective and efficient timetable. Therefore, this research project will focus on finding a solution to these 4 issues as they are the main and most commonly brought up by the commuters.

We can identify the main goal of the research project as the proposition of a new method of scheduling buses in the transport system so that the problems mentioned previously are addressed and minimized or eliminated. This would lead to a more efficient passenger transport service and will improve commuter satisfaction and productivity.

\section{Project Scope}
\label{section-ProjectScope}

\paragraph{ } The research will focus on improving the scheduling of buses so that it is made more efficient and reliable. The project will focus on the timetabling aspect of the problem of public transit scheduling, as opposed to network planning, vehicle and crew scheduling and rostering. The project will attempt to offer an effective timetabling methodology to address the prevalent issues.

The research will analyze the past work done on bus and transport scheduling in Sri Lanka as well as in other countries. A new methodology will then be proposed (via headway manipulation) to schedule buses taking into account passenger demand, origin-destination patterns and vehicle location data.

The research will then attempt to validate the hypothesis by testing it on a bus route in the Western Province. Data will be collected via quantitative surveys and be analyzed to identify if the new strategy can improve the service. Service metrics such as passenger trip times and passenger counts in buses will be used to identify if the proposed strategy can show an improvement in the bus service.

\newpage

\section{Outline of Thesis}
\label{section-OutlineOfThesis}

Beyond this point, the thesis document will be structured as follows. Chapter 2 will provide a comprehensive review of the literature regarding timetabling, scheduling of buses and optimization of schedules. 

Chapter 3 will discuss and propose a methodology for scheduling buses to improve efficiency in the bus service and minimize or eliminate the problems discussed earlier in this chapter. Chapter 4 will provide information on the data gathering techniques used as well as an overview of the data collected.

Chapter 5 will discuss the results of the gathered data and its’ implications vis-\`{a}-vis the method proposed in Chapter 3. Finally in Chapter 6, the conclusions of this research project will be provided and possible future work will be discussed.

\newpage

\chapter {Literature Review}
\label{chapter-LitReview}

\paragraph{ } In the previous chapter, we analyzed the background and current state of the private bus service in the Western Province. The chapter also included a thorough explanation of the current bus scheduling process in the WP RPTA. A justification of the reasons for this research project was also provided in the chapter.  We shall now take a look at the theoretical aspects of bus scheduling and the literature involved with it.

\section*{Bus Scheduling and Dispatching}

\paragraph{ } The scheduling process of a public transport company generally comprises of 4 main steps, which are \textit{timetabling}, \textit{vehicle scheduling}, \textit{crew scheduling}, and \textit{crew rostering}. The transport mode could be a bus, tram, metro, train or airline. Figure~\ref{image-genericSchedulingProcess} illustrates the flow of the traditional scheduling process. It is carried out as is or the crew scheduling is carried out prior to the vehicle scheduling in certain instances. However, in recent times, the industry has begun adopting an integrated approach of carrying out the vehicle and crew scheduling together. This is also discussed later in the chapter \citep{Huisman2004}.

\begin {figure} [h!]
\centering
\includegraphics {genericSchedulingProcess}
\caption [Scheduling Process of a Public Transport Company] {Scheduling Process of a Public Transport Company. Source: \citep{Huisman2004}}
\label {image-genericSchedulingProcess}
\end {figure}

Decisions about which routes or lines to operate and how frequently, are inputs for the operational planning process. They can be either given by the marketing department of the company or determined by (local, regional or national) authorities. Furthermore, the travel times between various points on the route are assumed to be known. Based on the lines and frequencies, timetables are determined resulting in trips with corresponding start and end locations and times.

The second planning process is vehicle scheduling which consists of assigning vehicles to trips, resulting in a schedule for each vehicle. The vehicles, which are not in use for some time, are parked in a depot. A schedule for a vehicle is split into several vehicle blocks, where a new vehicle block starts at each departure from the depot. On such a block a sequence of tasks can be defined, where each task needs to be assigned to a working period for one crew (a duty) in the crew scheduling process. The feasibility of a duty is dependent on a set of collective agreements and labor rules that refer to sufficient rest time etcetera. Crew scheduling is short term crew planning (one day), i.e. assigning crew duties to tasks on each specific day, while the crew rostering process is long term crew planning (e.g. half a year) for constructing rosters from the crew duties.

\section{Timetabling}

\paragraph{ } Timetabling is the process of determining how frequently buses must operate on routes based on passenger demand and the operational plan of the respective authority and generating corresponding start and end locations and times of trips, based on these frequencies. Furthermore travel times between various points in routes, lengths of routes are assumed to be known \citep{Huisman2004}.

The main consideration in the process of timetabling is the output of headways, or the time between consecutive buses. Currently, average optimal headways are calculated in Sri Lanka and used in the scheduling process. An explanation of this process has been given in the previous chapter.

Let us take a look at previous research done into this area. \citet{Newell1971} provides the basic dispatching policy for a transportation route. This work has been the basis for many future work including \citet{Kumarage2007} research paper on formulating an optimal bus dispatching policy under variable demand over time and route length as is the case in Sri Lanka. The paper considered a method that was an extension to \citet{Newell1971}'s Optimal Dispatching Policy to determine a fleet size and dispatching rate based on both the operator's cost and user's cost including the disutility of standing, in order to arrive at a global cost optimum.

\citet{Riano2004} propose a stochastic model for Bus Dispatching that uses a linear programming model so that it is solvable easily and can be used for other modes of mass transit as well. A feature of this model is that it should be applied to modes of mass transit where the frequency is high enough so that users do not need to know the
schedule in advance (such as the one in Sri Lanka) The solution technique is based on a novel Transient Little Law.

Alternatively, \citet{Daganzo2008} describes an adaptive control scheme to mitigate the problem Bus Bunching in a Public Bus Transport system. This is a problem closely related to timetabling and scheduling and thus has been included in this Literature Review. Bus schedules cannot be easily maintained on busy lines with short headways. Experience shows that buses offering this type of service usually arrive irregularly at their stops, often in bunches. Although authorities build slack into their schedules to alleviate this problem, if necessary holding buses at control points to stay on schedule, their attempts often fail because practical amounts of slack cannot prevent large localized disruptions from spreading system-wide. 

The proposed scheme dynamically determines bus holding times at a route’s control points based on real-time headway information. The method requires less slack than the conventional, schedule-based approach to produce headways within a given tolerance. This allows buses to travel faster, reducing in-vehicle passenger delay and increasing bus productivity. One disadvantage of this method when thinking in terms of a Sri Lankan context is that it requires real-time information which cannot be obtained currently in the Bus Transport System in Colombo.

\citet{Xuan2011} look at the timetabling problem and its effect on schedule reliability a little differently. They study several Dynamic Holding Strategies that use the current state of all buses, as well as a virtual schedule. The virtual schedule is introduced whether the system is run with a 
published schedule or not. Through their research, it was found that through a dynamic holding strategy buses can both closely adhere to schedule and maintain regular headways without too much slack. This in turn improves Schedule Reliability and Commercial Speed of the buses.

\citet{Ceder2009} in his paper, put forth a methodology framework with developed algorithms for the derivation of vehicle departure times (timetable) with either even headways or even average passenger loads. The latter would be ideal for a situation like in Sri Lanka where overcrowding is a major complaint among the commuters.

\citet{Qian2013}, in their paper studied the Optimizing Mathematical Model of Bus Departure Interval and its Solution, and then worked out the best Bus Departure Interval (i.e.: the Headway). They established the optimizing mathematical model of bus departure interval, which took crowding cost of passengers into account, including passengers’ on-the-bus time cost, passengers’ crowding cost and bus company cost. The paper used a genetic algorithm to solve the problem. Reasonable bus departure intervals were obtained quickly by this method.

Furthermore, \citet{Fu2003} presented a new transit operating strategy in which service vehicles operate in pairs with the lead vehicle providing an all-stop local service and the following vehicle being allowed to skip some stops as an express service to address the bus dispatching and timetabling problem. The underlying scheduling problem is formulated as a nonlinear integer programming problem with the objective of minimizing the total costs for both operators and passengers. This could possibly be adapted into the Sri Lankan context as a viable alternative to the current scheduling methodology.

Finally, \citet{Sun2008} studied the headway optimization and scheduling combination of Bus Rapid Transit (BRT) vehicles. A model was proposed to minimize passengers' travel costs and vehicles' operation cost, and constraints included passenger volume, time, and frequency. The scheduling combination was composed by Normal, Zone, and Express scheduling. The model was solved by a genetic algorithm of variable-length coding. The result of the numerical case showed that the optimization results can save 69.92\% of the cost. A sensitivity analysis that was carried out showed that, under higher traffic volume or lower speed, the travel cost can be reduced through reasonable scheduling combination. The method was, therefore, proven scientifically and is feasible. A similar scheduling method of using Normal, Zone and Express Scheduling could possibly be implemented in Sri Lanka.

\section{Vehicle Scheduling}

\paragraph{ } Vehicle scheduling is defined as the process of assigning vehicles, to a set of predetermined trips, with fixed starting and ending times, while minimizing capital and operational costs. According to \citet{Freling2003}, the main objective of this step in the Scheduling Process is keeping the operational and capital costs of vehicles to a minimum.

The Vehicle Scheduling problem can be thought of through two perspectives. They are the Single-Depot Vehicle Scheduling Problem (SDVSP) and the Multiple-Depot Vehicle Scheduling Problem (MDVSP). As noticeable by their names, the difference is the number of depots that each Vehicle is assigned to. In the former, a given vehicle is assigned to a single depot while the latter situation assumes that a given vehicle is assigned to multiple depots. The SDVSP further assumes that all vehicles are identical and there are no time constraints \citep{Huisman2004}. Of these 2 approaches, only the SDVSP applies to the Sri Lankan context as each bus (and its crew) is issued a permit to ply on one route. Therefore, \citet{Freling2003} defines the Single-Depot Vehicle Scheduling Problem as follows.

“Given a depot at location \textit{d} and \textit{n} trips from locations \textit{b}$_i$ to \textit{e}$_i$, with corresponding fixed times \textit{bt}$_i$ and \textit{et}$_i$ (\textit{i}=1,...,\textit{n}),  and  given  the  travelling  times  between  all  pairs (\textit{d}, \textit{b}$_i$), (\textit{b}$_i$, \textit{e}$_i$), (\textit{e}$_i$, \textit{b}$_j$) and (\textit{e}$_i$, \textit{d}), find a feasible \textit{minimum cost} schedule for the vehicles, such that all trips are covered by a vehicle. Each trip has to be entirely serviced by one vehicle and trips serviced by the same vehicle are linked by \textit{deadheading trips} (\textit{dh}-trips).  These are trips without serving passengers (pairs (\textit{d}, \textit{b}$_i$), (\textit{e}$_i$, \textit{b}$_j$) and (\textit{e}$_i$, \textit{d})), consisting of travel time (vehicle deadheading) and/or \textit{idle time} (vehicle waiting time). A schedule for a vehicle is composed of vehicle \textit{blocks}, where each block is a departure from the depot, the service of a sequence of trips and the return to the depot. The cost function is a combination of vehicle capital (fixed) and/or operational (variable) cost. The capital cost is often such that the number of vehicles will be minimized, while the operational cost is often a combination of vehicle travel and idle time”.

In the Multiple-Depot Vehicle Scheduling Problem (MDVSP), buses are dispatched from several depots and total vehicles costs have to be minimized subject to the following constraints \citep{Huisman2004}

\begin {itemize}
\item Every vehicle is associated with a single depot; 
\item Every trip has to be assigned to exactly one vehicle; 
\item Some trips have to be assigned to vehicles from a certain subset of depots.
\end {itemize}

\section {Crew Scheduling}

\paragraph{ } The next step in the traditional scheduling process is the Crew Scheduling.  The problem is defined as follows: “The Crew Scheduling Problem (CSP) deals with assigning tasks to duties such that each task is performed; each duty is feasible with respect to a set of working rules and the total costs of the duties are minimized”. 

According to \citep{Huisman2004}, the only requirement with relation to the feasibility of a duty is its length or the working time in that duty, respectively. In most literature, the CSP is formulated as a set partitioning or covering problem and solved with a column generation approach.

\section {Integrated Vehicle and Crew Scheduling}

\paragraph{ } As mentioned previously in this chapter, the industry has begun using Integrated Vehicle and Crew Scheduling in recent times. This is due to logical as well as financial reasons. The airline industry uses this methodology as a crew has to be scheduled along with an airline trip for obvious reasons.

The Integrated Vehicle and Crew Scheduling Problem (Integrated VCSP or IVCSP) is analyzed by \citet{Huisman2004}. Accordingly, the Integrated Vehicle and Crew Scheduling Problem is defined as follows. “Given a set of service requirements or trips within a fixed planning horizon, find a minimum cost schedule for the vehicles and the crews, such that both the vehicle and the crew schedules are feasible and mutually compatible. Each trip has fixed starting and ending times, and the travelling times between all pairs of locations are known. A vehicle schedule is feasible if (1) each trip is assigned to a vehicle, and (2) each vehicle performs a feasible sequence of trips, where a sequence of trips is feasible if it is feasible for a vehicle to execute each pair of consecutive trips in the sequence”.

The author provides a very comprehensive look into the problem and presents a methodology to solve integrations of both the SDVSP and the MDVSP. The Integrated VCSP is also discussed by \citet{Freling2000} and \citet{Wren1997}.

\section {Crew Rostering}

The Crew Rostering Problem (CRP) aims at determining an optimal sequencing of a given set of duties into rosters satisfying operational constraints deriving from union contracts and company regulations. In the public bus transport industry, the roster is used to evenly distribute the workload among the crew \citep{Caprara1995}. Further studies were done by \citep{Kharraziha2003} and \citep{Tian2012}. The Integrated Crew Rostering Problem was examined in depth by \citep{Valdes2010} and \citep{Xie2012}.

This literature review looked at the different steps involved in the bus scheduling process. It also reviewed the literature related to each step. This literature is what will act as reference material as we attempt to present a new model of scheduling buses. The next chapter will detail the research methodology and the proposed model for scheduling buses in the Western Province.

\newpage

\chapter{Research Methodology}
\label{chapter-ResearchMethodology}

\paragraph{ } In Chapter~\ref{chapter-LitReview} we discussed the theoretical basis of scheduling and the planning process of a public transport system. We also analyzed the current literature related to the various stages of the planning process (i.e.:timetabling, vehicle and crew scheduling and crew rostering). Furthermore, in Chapter~\ref{chapter-Introduction} we discussed the background of the present system and the problems associated with the current bus scheduling process as well as the bus transport system as a whole. This chapter will detail a methodology to dispatch buses so as to minimize/eliminate the mentioned shortfalls in the system.

\section{Proposed Bus Dispatching Methodology}

\paragraph{ } The methodology proposed in this thesis builds on the work of \citet{Ceder2009} and a few others. The objective of the methodology is to find out a suitable dispatching method for buses in each of the 450+ routes in the Western Province Private Bus Transport Service. His work involved 3 procedures, where the first proposed to dispatch buses with even headways disregarding the passenger demand (a.k.a the Load Factor in the bus a.k.a the number of people on board the bus). The second procedure proposed a method to dispatch buses so that the passenger demand (Load Factor) is minimized at the stop where the passenger demand is highest (for example, if in a route with stops at A, B, C and passenger demand is at its peak at B, this procedure attempts to dispatch buses so that the Load Factor at B is minimized thereby improving the comfort levels of travelling commuters).

The third procedure proposed a method of dispatching buses where the passenger demand at all the stops were controlled at a predefined value. This means the passenger loads are more evenly spread out among the buses thereby minimizing overcrowding and increasing commuter comfort levels immensely. The purpose of the procedure is to derive the bus timetable provided that in an average sense all buses will have even Loads (equal to the Desired Occupancy or d$_j$, where j is the time period under observation. for example: 1st hour: j=1, d$_1$=Desired Occupancy for 1st hour; 2nd hour: j=2, d$_2$=Desired Occupancy for 2nd hour etc.) at the Max Load stop of each bus. That is, for a given time period each bus may have a different Max Load point across the entire bus route with a different Observed Average Load. The objective set forth is to change the departure times such that all Observed Average Max Loads will be the same and equal to d$_j$ during all j. Certainly the adjustments in the timetable are not intended for highly frequent urban services where the headway is less than say, 10 minutes, or an hourly frequency of about 6 vehicles or more. Behind this procedure is the notion that passenger overcrowding situations (loads greater than d$_j$) should be avoided.

The proposed proposed system would identify the stakeholders of the private bus service, take into account the factors that affect each of them and evaluate the factors according to several dispatching methods to identify which methodology best suits the route in question. In effect, the system would evaluate the feasibility of each methodology for each route and output the most feasible method. Please refer to Figure~\ref{image-feasibilityMatrix}.

\begin {figure} [h!]
\centering
\includegraphics[scale=0.6]{feasibilityMatrix}
\caption [Feasibility Matrix] {Feasibility Matrix}
\label {image-feasibilityMatrix}
\end {figure}

As in the Figure~\ref{image-feasibilityMatrix}, the suitable methodology will be the output for every route.

\newpage

\chapter {Data Gathering and Analysis}
\label{chapter-DataGatheringAndAnalysis}

\section{Data Gathering Technique}

\paragraph{ } In Chapter~\ref{chapter-DataGatheringAndAnalysis} we discussed the proposed bus dispatching methodology to solve the problems mentioned in Section~\ref{section-ExistingProblems}. This chapter will detail the Data Gathering procedure as well as some research into the problems related Data Gathering and the Private Bus Transport System in the Western Province (WP). It will also provide a sample of the data that was gathered during an initial pilot survey. The complete set of data that was gathered is listed in Appendix~\ref{appendix-CompleteSetOfData}.

The bus route chosen for the study is the 177 bus route that runs between Kolpetty and Kaduwela. The route segment between Kolpetty and Rajagiriya is taken into consideration when collecting the data. It is a high value route and is graded as "A+" in a scale of A+, A, B, C by the WP RPTA based on the passenger demand and importance of the route. Additionally, it has buses that are consistently overcrowded and commuters complain about it frequently. Buses on the route are also prone to loitering and many complaints are received regarding this issue making it a perfect candidate to test our hypothesis.

There are 49 buses in in the normal fleet in total with 35 in daily operation. All buses have the capacity to carry around 45 passengers (seated) on average. The route also has Air Conditioned (AC) buses but we will not be considering the AC buses in our data gathering.

Data is collected by physically traveling in the bus at the required times and manually counting the number of passengers that get on and off the bus. This provides information about the passenger demand and load situations in the bus. The time at which the bus arrives and departs each bus stop is also observed and recorded. The time spent at each bus stop can be calculated using this data. If the bus loiters at a certain bus stop, the amount of time the driver spends loitering is also recorded.

\section{Loitering} Loitering refers to the time the bus lingers at the bus stop waiting for more passengers after the first group of passengers have alighted and boarded the bus. The time taken for commuters to get on and off the bus is not considered under the Loiter Time. A common complaint by commuters is that buses, after dropping off the passengers and allowing more to get in, idle at the bus stop usually until the next bus arrives. This leads to the people who are already in the bus waiting idly at the bus stop in a crowded bus senselessly. This action also picks up the passengers that are supposed to go in the later bus which leads to the later bus not having enough passengers to cover their costs and thereby they too loiter at the bus stop. This occurs continuously and in the end the commuters are kept waiting in the bus at the bus stand for no reason. Loitering is a major issue with the Private Bus Service and frequent complaints are made by commuters regarding it. It is detrimental to the proper dispatching of buses as it delays the commuters and makes it difficult for the schedulers to draw up correct timetables.

The data is gathered in this manual way because there is no automated data gathering mechanism currently in place for the private bus system in the WP.

\section{Automated Data Gathering}

\paragraph{ } While an automatic data gathering system would be immensely helpful to the scheduling unit of the WP RPTA, there is no such system in place currently. Such a system would aid in formulating proper timetables as well as monitoring and regulating the timetables and the buses for their adherence to said timetables. Proper and accurate data is very difficult to gather at the moment as all data gathering is done manually. This involves a huge amount of man-hours of work not forgetting the fact that the sample gathered is assumed to be representative of the whole system which might not always be the case. 

Furthermore, a route survey is only done when there is a huge change needed to the timetable; otherwise the timetable is a fixed entity. This is a very reactive stance to the situation which is the wrong policy to adopt as passenger demands vary and the service and the timetable needs to adjust accordingly. At present however, once the timetable is formulated and signed off by the higher authorities, it is not changed unless a significant number of people complain about. The service and timetable needs to be more proactive in order to offer a better service to the public.

An ideal Automatic Data Collection System (ADCS) would include provisions for Automatic Vehicle Location (AVL) data, Automatic Passenger Count (APC) data as well as the integration of an Automatic Fare Collection (AFC) System. Please refer to Figure~\ref{image-ADCS}. This would allow the schedulers to formulate much more accurate timetables taking into account the Passenger Demands (Load Factors) of the various buses at the various stops during various times of day. An Origin-Destination Matrix could also be obtained by analyzing this data to determine which segments of routes are most frequently used and therefore are more likely to be congested.

\begin {figure} [h!]
\centering
\includegraphics[scale=0.6]{ADCS}
\caption [An Automatic Data Collection System] {An Automatic Data Collection System}
\label {image-ADCS}
\end {figure}

An implementation of an Automatic Data Collection System (ADCS) could ultimately lead to the creation of Dynamic Timetables. This means that the buses could be scheduled using real-time passenger demand data as well as traffic information and thus would better suit the commuters. However, the current system is to do the scheduling process manually in a strictly procedural fashion.

An ADCS has been proposed in the past but has been shot down and discouraged by the bus owners and operators. The main reason that they give for this is that the system is too costly for them to implement and it brings no additional value to them. This is a misguided notion by the owners and operators as the system could benefit them immensely in the long run. Revenue leakage, which is one of the main grievances given by bus owners, could be curbed through an implementation of an Automatic Fare Collection (AFC) System. Furthermore, it would bring about a sense of accountability and transparency in the payment of fares of private buses. It would be a cost in the short-term, but considering the long-term benefits, the owners/operators should consider it an investment to make the system and service future-proof.

Furthermore, an Automatic Vehicle Location (AVL) system would allow commuters to know exactly where and when the next bus will be arriving providing more information and flexibility. Automatic Passenger Counting (APC) would give the schedulers enough information to ensure that timetables are working correctly and buses are dispatched to account for higher or lower passenger demand. The schedulers could also audit their timetables more frequently to ensure that schedules are kept current. It would also allow route monitors to ascertain if the owners/operators are overcrowding the buses causing discomfort to passengers. An Automatic Fare Collection (AFC) system would also bring great comfort to commuters who consistently complain about being overcharged or not receiving the proper balance money back from the conductor. An AFC would also greatly benefit the owners/operators because they could monitor and audit their income correctly and put paid to the revenue leakage that occurs consistently (according to the owners).

\subsection{Automatic Data Collection in the Inter-Provincial Service} 

\paragraph{ } An Automatic Data Collection System is in place at the moment in the Inter-Provincial buses. The National Transport Commission (NTC) has taken steps to implement a system to track the location of the buses via a GPS tracking device fixed onto them. However, out of the 3500+ buses in operation in the Inter-Provincial service, only around 700 have been fixed with the device. Information obtained from the NTC states that a further 1000 buses will be equipped with the device within this year.

Currently it only tracks the location and the speed of the bus. The system they have implemented consists of a device which is fixed on to the vehicle. This device tracks the location of the bus via GPS and relays the data back to the NTC servers via a GPRS connection. The device also contains an accelerometer to measure the speed of the vehicle. This allows the NTC to monitor the buses for possible speeding violations and reckless driving. Monitoring is done through a control center at the NTC head office in Narahenpita which is manned 24 hours a day. Apart from monitoring the buses, they provide information to the passengers and record complaints regarding errant bus operators.

Owners/operators can also choose to have a still camera fixed onto the bus as well. This is compulsory for Luxury buses and it allows the NTC monitor overcrowding of the buses. When a complaint is made to the NTC hotline, the personnel at the NTC control center can activate the camera and take a picture to decide if the bus is overcrowded or not. They can then instigate punishment proceedings for the bus involved.

The device costs around 45,000 rupees with the government subsidizing the cost. Eventually, the owners/operators have to pay only around 10,000 rupees. The objective of the NTC is to eventually have all the inter-provincial buses fitted with the device. This would allow the data to be collected automatically and that data would be used for the scheduling process and the formulation of timetables for the service.

\subsection{Automatic Vehicle Location (AVL)}

\paragraph{ } Vehicle Location data could be obtained through several methods. One of the main and most feasible methods would be to fix a GPS-based tracker on to the bus and monitoring its movement through a mapping software. This method, however, has a fairly large startup cost which has discouraged the stakeholders in the past. Alternatively, this could be achieved by crowdsourcing the Vehicle Location data through a mobile app. Automatic Vehicle Location (AVL) Data is very important as it reflects the current location of a bus and can be used to predict journey times and availability of buses for the commuters. A system could be built that uses AVL data to notify commuters of the current bus, the bus that is due to arrive next etc. This data is also very important in the scheduling stage of transport planning and could be used to formulate more efficient and effective timetables.

\subsection{Automatic Passenger Counting (APC)}

\paragraph{ } Automatic Passenger Counting (APC) is important in order to gauge the passenger demand or load factors in the buses. This data could be used to draw up origin-destination matrices in order to identify which segments of routes are most congested and which routes are not very congested at all. This all helps in the planning and dispatching of buses which eventually lead to a better schedule and a better service.

The APC data could be obtained from an Automated Fare Collection (AFC) System such as a contactless smart card payment method. The implementation of both could potentially be very useful for the commuters and owners/operators alike.

\subsection{Automatic Fare Collection (AFC)}

\paragraph{ } The implementation of an Automatic Fare Collection system has been in the works for some time and discussions with industry experts have revealed that a working system is available. However it is not being implemented as the government still has some regulatory and other issues to sort out first. The system involves a Prepaid Smart Card system where the Smart Card could be used for cashless payment of bus fares. Commuters could also recharge their cards when the stored value in the cards runs low (akin to reloading a prepaid mobile phone). This is just one of the methods that could be used to implement an automated fare collection system.

As its popularity grows, NFC or Near-Field Communication has garnered the attention of public transport systems around the world. It allows a user to use everyday devices (such as a mobile phone or an ATM card etc.) as a transaction medium. This could lead to enabling payment of bus fares with a mobile phone which is a very innovative and highly lucrative area to look into. It would bring comfort to passengers and allow owners/operators to receive the payment directly from the commuters. Therefore, the potential market and it's applications is vast. However, Sri Lanka does not have any such systems or parties interested in a system like this which is unfortunate. One could argue that before going to such technologically advanced lengths, the authorities should focus on a smaller system and implement one which is more appropriate to the Sri Lankan context which is a fair argument. The problem is that currently \textit{no system at all} exists, which is not a good situation for the private bus service and the country.

As mentioned previously, the NTC has implemented a GPS-based Vehicle Tracking system in the Inter-Provincial Bus Transport Service as an Automated Data Collection System. However, it only collectes the Vehicle Location data. The Passenger Counts data is not collected automatically and there is no system in place to do so. However, the Inter-Provincial Bus Service is more developed than the Intra-Provincial Service in the Western Province as the NTC has made it compulsory for Inter-Provincial buses to have Electronic Ticketing Machines (ETM). This means that at least the passenger counts data \textit{is} recorded. The data that is saved in the Ticketing Machines could be accessed later and the data be obtained for scheduling purposes.

\subsection{Gathering the Data} 

\paragraph{ } As mentioned previously, the current data gathering is done manually by the schedulers at the WP RPTA. The implementation of an ADCS has not been carried out yet due to the reasons mentioned earlier in the chapter. Therefore, the problem remains, how can the data be collected in order to aid in the scheduling and timetabling activities of the RPTA.

The data that is required for the schedulers is the Vehicle Location data and the Passenger Demand data. The former could be obtained by using just a few GPS tracking devices such as the ones the NTC uses in their system, which are fixed temporarily to a few buses. The problem the schedulers face currently is that they have to physically travel in the buses to collect the data. Alternatively, they could use a few (3 or 4) devices to temporarily track the selected buses for the purposes of the survey freeing the need for the schedulers to travel in the buses physically themselves and gather the data. The devices don't have to be fixed permanently onto the buses, just when the survey is being conducted so that the data can be gathered on its movements.

Gathering of the Passenger Demand data is a little more trickier. Passenger Demand is basically a reflection of the number of people that travel in a bus and details of where they get on and off the bus. This could be achieved by counting the number of tickets issued and analyzing the ticketing data. This method however, has a major drawback. It assumes that all commuters are issued a ticket which may not always be the case. Also, not all buses currently have Electronic Ticketing Machines which is an obstacle. The fact that it is not compulsory for the private buses in the WP plying on intra-provincial routes to have Electronic Ticketing Machines is also a major disadvantage to this method. In conclusion, it is clearly evident that reform and stricter regulation of the service is required in order to improve it.

\newpage

\section{Gathered Data}

\paragraph{ } Listed below is a sample of the data that was collected. The full list of data is available under Appendix~\ref{appendix-CompleteSetOfData}.

\begin{itemize}

\item Trip Number: 1
\begin{itemize}
\item Date: 23/5/2013
\item Departure Time: 15.40pm
\item Departure Place: Kolpetty
\item Table~\ref{table-trip1-BoardingAndAlighting} and~\ref{table-trip1-LoiterTime}
\end{itemize}
\begin{table}[h!]
\centering
\begin{tabular}{|l|r|r|r|r|}
\hline
Bus Stop & Boarded & Alighted & Net Gain & On Board \\
\hline
 & & & & 0 \\
Kolpetty Depot	&11	&0	&11	&11\\
Supermarket	&10	&0	&10	&21\\
Alwis Place	&7	&0	&7	&28\\
Library	&3	&6	&-3	&25\\
SLTA	&0	&2	&-2	&23\\
\rowcolor[gray]{0.7}
Museum	&1	&0	&1	&24\\
Nelum Pokuna	&2	&2	&0	&24\\
\rowcolor[gray]{0.7}
Alexandra Roundabout	&5	&0	&5	&29\\
Asha Central	&2	&1	&1	&30\\
Wijerama	&4	&0	&4	&34\\
Borella	&2	&5	&-3	&31\\
Devi Balika	&1	&1	&0	&31\\
Castle Street	&1	&0	&1	&32\\
Ayurveda	&2	&6	&-4	&28\\
Rajagiriya	&25	&5	&20	&48\\
\hline
\end{tabular}
\caption{Boarding And Alighting data for Trip 1}
\label{table-trip1-BoardingAndAlighting}
\end{table}

\begin{table}[h!]
\centering
\begin{tabular}{|l|r|r|r|}
\hline
Bus Stop & Arrival Time & Departure Time & Loiter Time (mins) \\
\hline
Kolpetty Depot	&	&1540	&0\\
Supermarket	&1544	&1545	&1\\
Alwis Place	&1546	&1546	&0\\
Library	&1548	&1549	&1\\
SLTA	&1550	&1551	&1\\
Nelum Pokuna	&1552	&1553	&1\\
Asha Central	&1556	&1556	&0\\
Wijerama	&1558	&1558	&0\\
Borella	&1608	&1609	&1\\
Devi Balika	&1611	&1611	&0\\
Castle Street	&1614	&1614	&0\\
Ayurveda	&1615	&1616	&1\\
Rajagiriya	&1618	&1624	&6\\
\hline
Total Loiter Time & & & 12 mins \\
Duration of Trip & & 44 mins & \\
\hline
\end{tabular}
\caption{Loiter Time Data for Trip 1}
\label{table-trip1-LoiterTime}
\end{table}

\begin {figure} [h!]
\centering
\includegraphics[scale=0.7]{passengerLoadData-Trip1}
\caption [Graph - Passenger Load Fluctuations - Trip 1] {Graph - Passenger Load Fluctuations - Trip 1}
\label {image-passengerLoadData-Trip1}
\end {figure}

\begin {figure} [h!]
\centering
\includegraphics[scale=1]{loiterTimeData-Trip1}
\caption [Graph - Bus Loiter Time Data - Trip 1] {Graph - Bus Loiter Time Data - Trip 1}
\label {image-loiterTimeData-Trip1}
\end {figure}

\end{itemize}

\appendix
\chapter{Complete Set of Gathered Data}
\label{appendix-CompleteSetOfData}

\paragraph{ } The fields that are highlighted in the tables are unauthorized stops at which the buses picked up or dropped off passengers.

\begin{enumerate}

\item Trip Number: 2
\begin{itemize}
\item Date: 23/5/2013
\item Departure Time: 17.00pm
\item Departure Place: Kolpetty
\item Table~\ref{table-trip2-BoardingAndAlighting} and~\ref{table-trip2-LoiterTime}
\end{itemize}

\item Trip Number: 3
\begin{itemize}
\item Date: 23/5/2013
\item Departure Time: 18.14pm
\item Departure Place: Kolpetty
\item Table~\ref{table-trip3-BoardingAndAlighting} and~\ref{table-trip3-LoiterTime}
\end{itemize}

\item Trip Number: 4
\begin{itemize}
\item Date: 23/5/2013
\item Departure Time: 16.28pm
\item Departure Place: Rajagiriya
\item Table~\ref{table-trip4-BoardingAndAlighting} and~\ref{table-trip4-LoiterTime}
\end{itemize}

\item Trip Number: 5
\begin{itemize}
\item Date: 23/5/2013
\item Departure Time: 17.43pm
\item Departure Place: Rajagiriya
\item Table~\ref{table-trip5-BoardingAndAlighting} and~\ref{table-trip5-LoiterTime}
\end{itemize}

\end{enumerate}

%data tables

\begin{table}
\centering
\begin{tabular}{|l|r|r|r|r|}
\hline
Bus Stop & Boarded & Alighted & Net Gain & On Board \\
\hline
 & & & & 0 \\
Kolpetty Depot	&21	&0	&21	&21\\
Supermarket	&7	&1	&6	&27\\
Alwis Place	&5	&0	&5	&32\\
\rowcolor[gray]{0.7}
Citibank	&2	&0	&2	&34\\
Library	&7	&2	&5	&39\\
SLTA	&0	&1	&-1	&38\\
Nelum Pokuna	&3	&0	&3	&41\\
\rowcolor[gray]{0.7}
Alexandra Roundabout	&3	&0	&3	&44\\
\rowcolor[gray]{0.7}
Saudi Embassy	&1	&0	&1	&45\\
Asha Central	&6	&0	&6	&51\\
Wijerama	&0	&0	&0	&51\\
Borella	&9	&2	&7	&58\\
Devi Balika	&5	&0	&5	&63\\
Castle Street	&0	&0	&0	&63\\
Ayurveda	&0	&9	&-9	&54\\
Rajagiriya	&11	&7	&4	&58\\
\hline
\end{tabular}
\caption{Boarding And Alighting data for Trip 2}
\label{table-trip2-BoardingAndAlighting}
\end{table}

\begin{table}
\centering
\begin{tabular}{|l|r|r|r|}
\hline
Bus Stop & Arrival Time & Departure Time & Loiter Time (mins) \\
\hline
Kolpetty Depot	&	&1703	&0\\
Supermarket	&1703	&1704	&1\\
Alwis Place	&1705	&1705	&0\\
Library	&1708	&1708	&0\\
SLTA	&1709	&1709	&0\\
Nelum Pokuna	&1711	&1712	&1\\
Asha Central	&1715	&1716	&1\\
Wijerama	&1717	&1718	&1\\
Borella	&1721	&1725	&4\\
Devi Balika	&1727	&1727	&0\\
Castle Street	&1729	&1729	&0\\
Ayurveda	&1731	&1738	&7\\
Rajagiriya	&1740	&1743	&3\\
\hline
Total Loiter Time & & & 18 mins \\
Duration of Trip & & 40 mins & \\
\hline
\end{tabular}
\caption{Loiter Time Data for Trip 2}
\label{table-trip2-LoiterTime}
\end{table}

\begin{table}
\centering
\begin{tabular}{|l|r|r|r|r|}
\hline
Bus Stop & Boarded & Alighted & Net Gain & On Board \\
\hline
 & & & & 0 \\
Kolpetty Depot	&37	&0	&37	&37\\
Supermarket	&10	&0	&10	&47\\
Alwis Place	&1	&0	&1	&48\\
Library	&4	&3	&1	&49\\
SLTA	&1	&0	&1	&50\\
\rowcolor[gray]{0.7}
Museum	&2	&0	&2	&52\\
Nelum Pokuna	&5	&2	&3	&55\\
Saudi Embassy	&0	&0	&0	&55\\
Asha Central	&3	&1	&2	&57\\
Wijerama	&0	&0	&0	&57\\
Borella	&10	&6	&4	&61\\
Devi Balika	&1	&0	&1	&62\\
Castle Street	&6	&2	&4	&66\\
Ayurveda	&3	&5	&-2	&64\\
Rajagiriya	&18	&12	&6	&70\\
\hline
\end{tabular}
\caption{Boarding And Alighting data for Trip 3}
\label{table-trip3-BoardingAndAlighting}
\end{table}

\begin{table}
\centering
\begin{tabular}{|l|r|r|r|}
\hline
Bus Stop & Arrival Time & Departure Time & Loiter Time (mins) \\
\hline
Kolpetty Depot	&	&1819	&0\\
Supermarket	&1820	&1822	&2\\
Alwis Place	&1823	&1824	&1\\
Library	&1826	&1826	&0\\
SLTA	&1828	&1828	&0\\
Nelum Pokuna	&1830	&1830	&0\\
Asha Central	&1835	&1835	&0\\
Wijerama	&1835	&1835	&0\\
Borella	&1837	&1838	&1\\
Devi Balika	&1841	&1841	&0\\
Castle Street	&1843	&1843	&0\\
Ayurveda	&1845	&1845	&0\\
Rajagiriya	&1847	&1853	&6\\
\hline
Total Loiter Time & & & 10 mins \\
Duration of Trip & & 34 mins & \\
\hline
\end{tabular}
\caption{Loiter Time Data for Trip 3}
\label{table-trip3-LoiterTime}
\end{table}

\begin{table}
\centering
\begin{tabular}{|l|r|r|r|r|}
\hline
Bus Stop & Boarded & Alighted & Net Gain & On Board \\
\hline
 & & & & 19 \\
Rajagiriya	&15	&1	&14	&33\\
Cotta Road	&2	&0	&2	&35\\
Ayurveda	&4	&3	&1	&36\\
Castle Street	&3	&1	&2	&38\\
Borella	&2	&3	&-1	&37\\
Horton Place	&2	&2	&0	&37\\
Wijerama	&0	&2	&-2	&35\\
Asha Central	&7	&0	&7	&42\\
Nelum Pokuna	&2	&4	&-2	&40\\
Library	&19	&0	&19	&59\\
Liberty	&2	&16	&-14	&45\\
Galle Road	&0	&1	&-1	&44\\
\rowcolor[gray]{0.7}
St. Anthony's Mw	&1	&2	&-1	&43\\
Kolpetty Depot	&0	&43	&-43	&0\\
\hline
\end{tabular}
\caption{Boarding And Alighting data for Trip 4}
\label{table-trip4-BoardingAndAlighting}
\end{table}

\begin{table}
\centering
\begin{tabular}{|l|r|r|r|}
\hline
Bus Stop & Arrival Time & Departure Time & Loiter Time (mins) \\
\hline
Rajagiriya	&1628	&1633	&5\\
Cotta road	&1634	&1634	&0\\
Ayurveda	&1635	&1636	&1\\
Castle street	&1638	&1639	&1\\
Borella	&1640	&1641	&1\\
Horton Place	&1642	&1642	&0\\
Wijerama	&1644	&1644	&0\\
Asha Central	&1647	&1647	&0\\
Nelum Pokuna	&1648	&1648	&0\\
Library	&1650	&1650	&0\\
Liberty	&1652	&1652	&0\\
Galle Road	&1654	&1654	&0\\
Kolpetty Depot	&1655	&	&0\\
\hline
Total Loiter Time & & & 8 mins \\
Duration of Trip & & 22 mins & \\
\hline
\end{tabular}
\caption{Loiter Time Data for Trip 4}
\label{table-trip4-LoiterTime}
\end{table}

\begin{table}
\centering
\begin{tabular}{|l|r|r|r|r|}
\hline
Bus Stop & Boarded & Alighted & Net Gain & On Board \\
\hline
 & & & & 30 \\
Rajagiriya	&5	&2	&3	&33\\
Cotta Road	&2	&0	&2	&35\\
Ayurveda	&0	&4	&-4	&31\\
Castle Street	&2	&0	&2	&33\\
Borella	&1	&6	&-5	&28\\
Horton Place	&3	&2	&1	&29\\
\rowcolor[gray]{0.7}
Chatz	&1	&0	&1	&30\\
Wijerama	&1	&0	&1	&31\\
Asha Central	&1	&1	&0	&31\\
Nelum Pokuna	&5	&3	&2	&33\\
Library	&4	&3	&1	&34\\
Liberty	&0	&8	&-8	&26\\
Galle Road	&0	&4	&-4	&22\\
\rowcolor[gray]{0.7}
St. Anthony's Mw	&4	&3	&1	&23\\
Kolpetty Depot	&0	&23	&-23	&0\\
\hline
\end{tabular}
\caption{Boarding And Alighting data for Trip 5}
\label{table-trip5-BoardingAndAlighting}
\end{table}

\begin{table}
\centering
\begin{tabular}{|l|r|r|r|}
\hline
Bus Stop & Arrival Time & Departure Time & Loiter Time (mins) \\
\hline
Rajagiriya	&1740	&1743	&3\\
Cotta Road	&1744	&1745	&1\\
Ayurveda	&1749	&1749	&0\\
Castle Street	&1752	&1752	&0\\
Borella	&1755	&1756	&1\\
Horton Place	&1757	&1758	&1\\
Wijerama	&1759	&1759	&0\\
Asha Central	&1800	&1800	&0\\
Nelum Pokuna	&1801	&1802	&1\\
Library	&1803	&1803	&0\\
Liberty	&1805	&1805	&0\\
Galle Road	&1808	&1808	&0\\
Kolpetty Depot	&1809	&	&0\\
\hline
Total Loiter Time & & & 7 mins \\
Duration of Trip & & 26 mins & \\
\hline
\end{tabular}
\caption{Loiter Time Data for Trip 5}
\label{table-trip5-LoiterTime}
\end{table}


\bibliography{FYP}
\end{document}  